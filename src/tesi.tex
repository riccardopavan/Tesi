        %%******************************************%%
        %%                                          %%
        %%        Modello di tesi di laurea         %%
        %%            di Andrea Giraldin            %%
        %%                                          %%
        %%             2 novembre 2012              %%
        %%                                          %%
        %%******************************************%%


% I seguenti commenti speciali impostano:
% 1. 
% 2. PDFLaTeX come motore di composizione;
% 3. tesi.tex come documento principale;
% 4. il controllo ortografico italiano per l'editor.

% !TEX encoding = UTF-8
% !TEX TS-program = pdflatex
% !TEX root = tesi.tex
% !TEX spellcheck = it-IT

% PDF/A filecontents
\RequirePackage{filecontents}
\begin{filecontents*}{\jobname.xmpdata}
  \Title{Analisi e sviluppo di una applicazione web per la schedulazione e rendicontazione delle attività aziendali interne}
  \Author{Riccardo Pavan}
  \Language{it-IT}
  \Subject{Questa tesi di laurea descrive il lavoro svolto e i risultati ottenuti durante il periodo di stage presso Wavelop Srls, con sede a Treviso. L'obiettivo del tirocinio era lo sviluppo di una applicazione web dedicata alla gestione delle attività aziendali interne, in particolare a mostrarle in formato sia tabellare che grafico, permettendo di filtrarle in vari modi. La tesi si propone inoltre di descrivere la metodologia di lavoro adottata e le tecnologie utilizzate per lo sviluppo della applicazione.}
  \Keywords{Applicazione web\sep Visualizzazione dati\sep React\sep Node.js \sep MongoDB}
\end{filecontents*}

\documentclass[10pt,                    % corpo del font principale
               a4paper,                 % carta A4
               twoside,                 % impagina per fronte-retro
               openright,               % inizio capitoli a destra
               english,                 
               italian,                 
               ]{book}    

%**************************************************************
% Importazione package
%************************************************************** 

\PassOptionsToPackage{dvipsnames}{xcolor} % colori PDF/A
\usepackage[table]{xcolor}
\usepackage{colorprofiles}
\usepackage{makecell}
\usepackage[a-2b,mathxmp]{pdfx}[2018/12/22]
                                        % configurazione PDF/A
                                        % validare in https://www.pdf-online.com/osa/validate.aspx

%\usepackage{amsmath,amssymb,amsthm}    % matematica

\usepackage[T1]{fontenc}                % codifica dei font:
                                        % NOTA BENE! richiede una distribuzione *completa* di LaTeX

\usepackage[utf8]{inputenc}             % codifica di input; anche [latin1] va bene
                                        % NOTA BENE! va accordata con le preferenze dell'editor

\usepackage[english, italian]{babel}    % per scrivere in italiano e in inglese;
                                        % l'ultima lingua (l'italiano) risulta predefinita

\usepackage{bookmark}                   % segnalibri

\usepackage{caption}                    % didascalie

\usepackage{chngpage,calc}              % centra il frontespizio

\usepackage{csquotes}                   % gestisce automaticamente i caratteri (")

\usepackage{emptypage}                  % pagine vuote senza testatina e piede di pagina

\usepackage{epigraph}			% per epigrafi

\usepackage{eurosym}                    % simbolo dell'euro

%\usepackage{indentfirst}               % rientra il primo paragrafo di ogni sezione

\usepackage{graphicx}                   % immagini

\usepackage{hyperref}                   % collegamenti ipertestuali

\usepackage[binding=5mm]{layaureo}      % margini ottimizzati per l'A4; rilegatura di 5 mm

\usepackage[]{listings}                   % codici

\usepackage{microtype}                  % microtipografia

\usepackage{mparhack,fixltx2e,relsize}  % finezze tipografiche

\usepackage{nameref}                    % visualizza nome dei riferimenti                                      
\usepackage[font=small]{quoting}        % citazioni

\usepackage{subfig}                     % sottofigure, sottotabelle

\usepackage[italian]{varioref}          % riferimenti completi della pagina

\usepackage{booktabs}                   % tabelle                                       
\usepackage{tabularx}                   % tabelle di larghezza prefissata                                    
\usepackage{longtable}                  % tabelle su più pagine                                        
\usepackage{ltxtable}                   % tabelle su più pagine e adattabili in larghezza

\usepackage[toc, acronym]{glossaries}   % glossario
                                        % per includerlo nel documento bisogna:
                                        % 1. compilare una prima volta tesi.tex;
                                        % 2. eseguire: makeindex -s tesi.ist -t tesi.glg -o tesi.gls tesi.glo
                                        % 3. eseguire: makeindex -s tesi.ist -t tesi.alg -o tesi.acr tesi.acn
                                        % 4. compilare due volte tesi.tex.

\usepackage[backend=biber,style=verbose-ibid,hyperref,backref, sorting=none]{biblatex}
                                        % eccellente pacchetto per la bibliografia; 
                                        % produce uno stile di citazione autore-anno; 
                                        % lo stile "numeric-comp" produce riferimenti numerici
                                        % per includerlo nel documento bisogna:
                                        % 1. compilare una prima volta tesi.tex;
                                        % 2. eseguire: biber tesi
                                        % 3. compilare ancora tesi.tex.

\input{tesi-config}                     % file con le impostazioni personali
\begin{document}
%**************************************************************
% Materiale iniziale
%**************************************************************
\frontmatter
\input{inizio-fine/frontespizio}
\input{inizio-fine/colophon}
% !TEX encoding = UTF-8
% !TEX TS-program = pdflatex
% !TEX root = ../tesi.tex

%**************************************************************
% Sommario
%**************************************************************
\cleardoublepage
\phantomsection
\pdfbookmark{Sommario}{Sommario}
\begingroup
\let\clearpage\relax
\let\cleardoublepage\relax
\let\cleardoublepage\relax

\chapter*{Sommario}

Questa tesi di laurea descrive il lavoro svolto e i risultati ottenuti durante il periodo di stage presso Wavelop Srls, con sede a Treviso.\\
L'obiettivo del tirocinio era lo sviluppo di una applicazione web dedicata alla gestione delle attività aziendali interne, in particolare a mostrarle in formato sia tabellare che grafico, permettendo di filtrarle in vari modi. \\
La tesi si propone inoltre di descrivere la metodologia di lavoro adottata e le tecnologie utilizzate per lo sviluppo della applicazione. 

%\vfill
%
%\selectlanguage{english}
%\pdfbookmark{Abstract}{Abstract}
%\chapter*{Abstract}
%
%\selectlanguage{italian}

\endgroup			

\vfill


% !TEX encoding = UTF-8
% !TEX TS-program = pdflatex
% !TEX root = ../tesi.tex

%**************************************************************
% Ringraziamenti
%**************************************************************
\cleardoublepage
\phantomsection
\pdfbookmark{Ringraziamenti}{ringraziamenti}

\bigskip

\begingroup
\let\clearpage\relax
\let\cleardoublepage\relax
\let\cleardoublepage\relax

\chapter*{Ringraziamenti}

\noindent \textit{Ringrazio il Prof. Davide Bresolin, relatore della mia tesi, per l'aiuto, i consigli e il supporto fornitomi durante la stesura del lavoro.}\\

\noindent \textit{Ringrazio i miei genitori e familiari per il sostegno, il grande aiuto e per essermi stati vicini in ogni momento durante gli anni di studio.}\\

\noindent \textit{Ringrazio i colleghi di Wavelop che mi hanno seguito durante lo stage e mi hanno assicurato un ottima esperienza di tirocinio.}\\

\noindent \textit{Ringrazio, infine, i miei amici universitari e non per essermi stati vicini durante questi anni e con cui ho condiviso bellissime esperienze.}\\
\bigskip

\noindent\textit{\myLocation, \myTime}
\hfill \myName

\endgroup


% !TEX encoding = UTF-8
% !TEX TS-program = pdflatex
% !TEX root = ../tesi.tex

%**************************************************************
% Indici
%**************************************************************
\cleardoublepage
\pdfbookmark{\contentsname}{tableofcontents}
\setcounter{tocdepth}{2}
{
    \hypersetup{linkcolor=black}
    \tableofcontents
}
%\markboth{\contentsname}{\contentsname} 
\clearpage

\begingroup 
    \let\clearpage\relax
    \let\cleardoublepage\relax
    \let\cleardoublepage\relax
    %*******************************************************
    % Elenco delle figure
    %*******************************************************    
    \phantomsection
    \pdfbookmark{\listfigurename}{lof}
    \listoffigures



    %*******************************************************
    % Elenco delle tabelle
    %*******************************************************
    \phantomsection
    \pdfbookmark{\listtablename}{lot}
    \listoftables
        


    \lstlistoflistings
\endgroup

\cleardoublepage

\cleardoublepage

%**************************************************************
% Materiale principale
%**************************************************************
\mainmatter
\chapter{L'azienda}
\label{cap:azienda}

\section{Descrizione generale}

\begin{center}
	\includegraphics[height = 4cm]{wavelop-logo.png}
\end{center}

\noindent Wavelop S.R.L.S. è una giovane azienda nata nel 2018 con sede a Treviso che si occupa di consulenza e di progetti web e mobile per piccole e medie imprese.\\
L'azienda conta circa 10 dipendenti e offre un ambiente lavorativo giovane e dinamico. Predilige inoltre lo smartworking e la flessibilità degli orari.\\
Dal 2020 propone progetti per studenti universitari che siano interessati in un tirocinio presso la loro azienda. 

(progetti anche per clienti reali, inserimento in un team di lavoro)

\section{Modello di svluppo}

Wavelop segue un modello di sviluppo \textit{agile} con metodologia \textit{scrum}. Di seguito le caratteristiche che contraddistinguono questo metodo di lavoro:

\begin{itemize}
  \item il lavoro viene diviso in archi temporali della durata di una settimana, denominati \textit{sprint};
  \item all'inizio di ogni \textit{sprint} si tiene lo \textit{sprint planning}, una riunione in cui si chiarisce quali obiettivi, espressi sotto forma di \textit{user stories}, si mira a portare a termine;
  \item alla fine di ogni \textit{sprint} si tiene invece lo \textit{sprint review}, una riunione in cui si discute cosa si è fatto durante lo \textit{sprint}, mostrando i risultati ottenuti, anche tramite brevi demo anche ai clienti;
  \item all'inizio di ogni giornata lavorativa si partecipa allo \textit{stand-up meeting}, una breve riunione della durata di circa 15 minuti, dove, a turno, ogni membro del team espone brevemente cosa ha fatto il giorno precedente e cosa intende fare oggi, oltre che mettere al corrente tutti di eventuali ostacoli incontrati nello sviluppo.
\end{itemize}

Adottare una metodologia di questo tipo permette una comunicazione costante fra i membri del team e i clienti che porta a un tracciamento chiaro e preciso dell'avanzamento del progetto e a una messa in luce immediata di eventuali incomprensioni, permettendo quindi delle correzioni tempestive, in accordo col cliente.
\chapter{Descrizione dello stage}
\label{cap:azienda}

\section{L'azienda}

\begin{center}
	\includegraphics[height = 4cm]{wavelop-logo.png}
\end{center}

\noindent Wavelop S.R.L.S. è un'azienda nata nel 2018 con sede a Treviso che si occupa di consulenza e di progetti web e mobile per piccole e medie imprese e che conta circa 10 dipendenti.\\
Dal 2020 propone progetti per studenti universitari interessati a un tirocinio formativo che verranno poi  inseriti all'interno di un team di lavoro presso l'azienda.

\section{Modello di svluppo}

Wavelop segue un modello di sviluppo \textit{agile} con metodologia \textit{scrum}. Di seguito le caratteristiche che contraddistinguono questo metodo di lavoro:

\begin{itemize}
  \item il lavoro viene diviso in archi temporali della durata di una settimana, denominati \textit{sprint};
  \item all'inizio di ogni \textit{sprint} si tiene lo \textit{sprint planning}, una riunione in cui si chiarisce quali obiettivi, espressi sotto forma di \textit{user stories}, si mira a portare a termine;
  \item alla conclusione di ogni \textit{sprint} si tiene invece lo \textit{sprint review}, una riunione in cui si discute cosa si è fatto durante lo \textit{sprint}, mostrando i risultati ottenuti, di solito tramite brevi demo;
  \item l'attività giornaliera inizia con la partecipazione allo \textit{stand-up meeting}, una breve riunione della durata di circa 15 minuti, dove, a turno, ogni membro del team espone brevemente cosa ha fatto il giorno precedente e cosa intende fare oggi, oltre che mettere al corrente tutti di eventuali ostacoli incontrati nello sviluppo.
\end{itemize}

\noindent Adottare una metodologia di questo tipo permette una comunicazione costante fra i membri del team e i clienti che porta a un tracciamento chiaro e preciso dell'avanzamento del progetto, mettendo subito in evidenza eventuali incomprensioni, permettendo quindi delle correzioni tempestive, in accordo col cliente.

\section{Il progetto di stage}
\label{sec:progettostage}
L'azienda ha già iniziato a sviluppare un'applcazione web, denominata \emph{Timesheet}, destinata a uso interno, al fine di rendicontare e pianificare le attività dei vari componenti dell'azienda. \\
In particolare è già implementata la funzionalità di inserire attività singole nel database. \\
Manca, invece, la possibilità di mostrare, filtrandole a seconda della volontà dell'utente, le varie attività memorizzate, ottenendo quindi dati utili come la quantità di ore dedicate a un cliente in un determinato arco di tempo. \\
Queste funzionalità dovranno essere raggruppate nella sezione \emph{Reports} dell'applicazione. \\
Lo scopo finale del progetto è quindi quello di completare la sezione \emph{Reports}, sia per quanto riguarda la componente di frontend che per quella di backend.

\section{Requisiti}
\label{sec:requisiti}
Nel piano di lavoro sono stati redatti requisiti specifici che dovranno essere soddisfatti nel periodo di stage. Essi sono categorizzati in base alla importanza che ricoprono:


\paragraph{Obbligatori} Requisiti ad alta priorità, la cui importanza è primaria per la riuscita del progetto:

\begin{itemize}
  \item apprendimento delle tecnologie di sviluppo come React e Node.js e per il versionamento del progetto come git;
  \item gestione filtri avanzati di ricerca delle attività;
  \item visualizzazione a tabella delle attività filtrate;
  \item generazione file CSV delle attività filtrate.
\end{itemize}

\paragraph{Desiderabili} Requisiti a media priorità, non necessari per il completamento dello stage ma che comunque, se soddisfatti, contribuiscono notevolmente alla buona riuscita del progetto: 

\begin{itemize}
  \item generazione file PDF delle attività filtrate;
  \item visualizzazione attività tramite grafico;
  \item salvataggio preset filtri per ricerche future.
\end{itemize} 

\paragraph{Facoltativi} Requisiti a bassa priorità, portano un valore aggiunto allo stage:

\begin{itemize}
  \item visualizzazione widget laterale con statistiche relative all'utente;
  \item gestione responsive della piattaforma.
\end{itemize}

\section{Pianificazione}

Lo stage ha avuto luogo nel periodo di tempo dal 25 Luglio 2022 al 16 Settembre 2022, per un totale di 312 ore.
Sono state inoltre definite delle attività specifiche per ogni settimana di stage, rappresentate di seguito:

\paragraph{Prima settimana - Dal 25 al 29 Luglio}

\begin{itemize}
  \item discussione requisiti relativi al sistema da sviluppare con le persone coinvolte nel progetto;
  \item introduzione alla cultura aziendale;
  \item formazione sulle tecnologie adottate;
  \item prendere confidenza con la struttura già esistente e assegnazione degli strumenti necessari;
  \item analisi dei requisiti.
\end{itemize}

\noindent\textbf{Seconda settimana - Dall'1 al 5 Agosto}

\begin{itemize}
  \item analisi dei requisiti;
  \item progettazione architetturale.
\end{itemize}

\noindent\textbf{Terza settimana - Dall'8 al 13 Agosto}

\begin{itemize}
  \item progettazione architetturale;
  \item analisi e definizione user stories per lo sprint successivo.
\end{itemize}

\noindent\textbf{Dalla quarta alla ottava settimana - Dal 16 Agosto al 16 Settembre}

\begin{itemize}
  \item sviluppo delle user stories assegnate;
  \item sprint review con il referente e le altre persone coinvolte nel progetto;
  \item analisi e definizione user stories per lo sprint successivo.
\end{itemize}

\noindent Inoltre, nell'ultima settimana, è previsto un collaudo finale di ciò che è stato fatto.

\section{Tecnologie utilizzate}

È stato utilizzato lo stack tecnologico tipicamente scelto dall'azienda, composto da: \emph{Ract}, \emph{Node.js} e \emph{MongoDB}, di seguito descritte nel dettaglio, una per una.

\subsection{React}

React\footcite{site:react} è una libreria Javascript open-source component-based per creare interfacce utente mantenuta da \emph{Meta}. Adotta l'utilizzo di una sintassi che estende il linguaggio Javascript: il \emph{Javascript Synstax Extension} (JSX), che rende semplice e intuitivo creare i componenti che formano la pagina, come mostrato nell'esempio di codice \ref{code:react} .\\
\begin{code}[frame=tb, label={code:react}, caption={Esempio di utlizzo di codice JSX}]\\  
1  const name = 'Riccardo Pavan';
2
3  const ShowName = () => {
4    return (
5      <div>Mi chiamo {name}</div>
6    );
7  };

\end{code}

Un'altra particolarità di React sono gli \emph{hooks}, particolari funzioni che permettono di "agganciarsi" a varie funzionalità di \emph{lifecycle} che React offre senza dovere usare classi.

\subsection{Node.Js}

Node.Js\footcite{site:node.js} è un \emph{runtime environment} open-source basato su Javascript.\\
La sua architettura orientata agli eventi permette alle operazioni di input/output di essere asincrone ed è consona soprattutto per applicazioni scalabili e che necessitano di un grande \emph{throughput} .\\
L'utilizzo di Node.Js permette di utilizzare Javascript anche lato server, permettendo quindi di usarlo come unico linguaggio nello sviluppo di un'applicazione web, dando vita al paradigma detto \emph{Javascript everywhere}.\\

\subsection{MongoDB}

MongoDB\footcite{site:mongo} è un \emph{database management system} open-source non relazionale, orientato ai documenti, questi ultimi in formato BSON, simile a JSON.\\
La struttura non relazionale permette una maggiore libertà nello strutturare il database, utile quando si ha a che fare con molti dati molto diversi fra loro e che tendono a cambiare nel tempo.             
% !TEX encoding = UTF-8
% !TEX TS-program = pdflatex
% !TEX root = ../tesi.tex

%**************************************************************
\chapter{Sprint 1}
\label{cap:sprint1}
%**************************************************************

Il primo sprint è stato dedicato principalmente ad ambientarsi al clima aziendale e alle tecnologie che verranno utilizzate più avanti. \\
Ho quindi conosciuto gli altri membri del team e mi è stato fornito materiale da studiare inerente ai seguenti argomenti:
\begin{itemize}
  \item metodologia agile e SCRUM, informazioni su termini come user stories ed epics;
  \item workflow da utilizzare nel repository condiviso, convenzioni su nomine dei commit;
  \item riunioni di routine da svolgere quotidianamente o settimanalmente;
  \item Javascript, React, CSS, Node.JS, MongoDB e porgrammazione asincrona; 
\end{itemize}

\noindent Tutta la documentazione fornita si è rivelata essere pienamente comprensiva e utile, soprattutto per quanto riguarda le tecnologie, buona parte delle quali non avevo mai usato. \\\\
Oltre a formarmi sugli argomenti precedentemente elencati, ho anche impostato l'ambiente di lavoro necessario sia a far girare la app che a svilupparla. \\
In particolare ho utilizzato i seguenti strumenti:
\begin{itemize}
  \item Docker, per poter utilizzare la immagine del database della applicazione;
  \item Studio 3T, una GUI per interfacciarsi a database MongoDB.
\end{itemize}

\noindent Tramite riga di comando potevo poi far girare sia la parte client che la parte server in locale, permettendomi di modificare codice e avere feedback visivo immediato. \\\\
\noindent Infine ho redatto una bozza di un documento atto a elencare le user stories per avere un'idea più chiara di ciò che avrei dovuto fare negli sprint successivi.\\
Seppure il docmento non fosse ancora completo è risultato subito chiaro che sarebbe stata necessaria una libreria per creare grafici. Ho quindi dedicato l'ultimo giorno della settimana a ricercare delle librerie Javascript per grafici, evidenziandone vantaggi e difetti per potere poi giungere a una decisione con gli altri membri del team.             
% !TEX encoding = UTF-8
% !TEX TS-program = pdflatex
% !TEX root = ../tesi.tex

%**************************************************************
\chapter{Sprint 2}
\label{cap:sprint2}
%**************************************************************

Durante il secondo sprint ho redatto la versione conclusiva del documento relativo alle user stories, approvato poi dal mio relatore esterno.

\noindent Ogni user story del documento presenta i seguenti elementi:
\begin{itemize}
  \item nome della useer story;
  \item un numero intero positivo che indica il "peso" in termini temporali che ci si aspetta che la user story abbia. Questi numeri seguono l'andamento della serie di Fibonacci e sono stati utli per allocare meglio le varie user stories all'interno degli sprint; 
  \item breve descrizione di ciò che si deve ottenere una volta conclusa la user story, dal punto di vista dell'utente;
  \item elenco di task, suddivisi tra frontend e backend, che dicono in modo più speccifico e tecnico cosa fare per concludere la user story;
\end{itemize}

\noindent All'inizio di ogni capitolo relativo a uno sprint indicherò le user stories su cui andrò a lavorare.

Ho inoltre creato dei diagrammi UML dei casi d'uso. (Li metto qua?)

Per concludere la settimana ho lavorato a dei piccoli bugfix e features, anche per familiarizzare con la web app e i pattern utilizzati.
In particolare ho: 
\begin{itemize}
  \item sistemato un problema con la immagine nella pagina di login, che non si rimpiccioliva nel caso la finestra facesse lo stesso;
  \item fatto sì che, nella pagina utente, nel caso quest'ultimo non la avesse inserita, venissero mostate le iniziali dell'utente invece che nulla;
  \item implementato una finestra di dialogo che semplicemente chiede di confermare o annullare un'azione.
\end{itemize}

Seppure i primi due problemi fossero molto semplici e risolvibili principalmente con del codice CSS, creare la finestra di dialogo mi è stato utile sia a familiarizzare con React, che non avevo mai utilizzato, ma anche a capire meglio come fosse strutturata la applicazione.
             
% !TEX encoding = UTF-8
% !TEX TS-program = pdflatex
% !TEX root = ../tesi.tex

%**************************************************************
\section{Sprint 3}
\label{sec:sprint3}
%**************************************************************
\subsection{User stories assegnate}
Anche se inizialmente era un'attività prevista per la quarta settimana ho iniziato a lavorare effettivamente al progetto durante il terzo sprint. \\
Utilizzando il documento precedentemente menzionato previste mi sono state assegnate le user stories da completare durante la settimana.
\noindent Ognuna di esse presenta i seguenti elementi:
\begin{itemize}
  \item nome della user story;
  \item un numero intero positivo che indica il "peso" in termini temporali che ci si aspetta che la user story abbia. Questi numeri seguono l'andamento della serie di Fibonacci e sono stati utili per allocare meglio le varie user stories all'interno degli sprint. Questi numeri vanno solitamente da 1 (task da meno di un'ora) a non più di 13 (task per il quale può essere necessaria fino a una settimana); 
  \item breve descrizione di ciò che si deve ottenere una volta conclusa la user story, dal punto di vista dell'utente;
  \item elenco di task, suddivisi tra frontend e backend, che dicono in modo più specifico e tecnico cosa fare per concludere la user story;
\end{itemize}
I quali verranno presentati nella forma:

\paragraph{Nome user story (Tempo previsto per concludere la user story)}\mbox{} \\[\baselineskip]
\noindent Descrizione dela user story. \\

\noindent Tasks:

\begin{itemize}
  \item Task1;
  \begin{itemize}
    \item subTask1;
  \end{itemize}
\end{itemize}
\vspace*{5pt}
\paragraph{Visualizzazione tabella (5)}\mbox{} \\[\baselineskip]
\noindent Come utente autenticato che si trova nella sezione “Reports”, voglio poter vedere la tabella riassuntiva delle ore spese. \\

\noindent Tasks:

\begin{itemize}
  \item implementare il componente principale della pagina Reports;
  \item implementare il componente della tabella, anche per quanto riguarda il suo stile;
  \item implementare la chiamata al backend per richiedere i dati da mostrare nella tabella;
  \item implementare la logica di query nel backend per soddisfare la richiesta dati.
\end{itemize}

\paragraph{Paginazione tabella (5)}\mbox{} \\[\baselineskip]
\noindent Come utente autenticato che si trova nella sezione “Reports”, voglio avere la tabella paginata, in modo da non avere troppe righe nella stessa schermata. \\

\noindent Tasks:

\begin{itemize}
  \item modificare la chiamata al backend già implementata per paginare i dati;
  \item aggiungere bottoni per passare da una pagina all'altra della tabella;
  \item aggiungere opzione per decidere quante righe includere in una pagina.
\end{itemize}

\subsection{Visualizzazione tabella}
\noindent Innanzitutto ho dovuto aggiungere all'applicazione la sezione Reports in quanto, seppure fosse già presente la tab per raggiungerla, non c'erano nè il percorso nè il componente della pagina.
A tal scopo ho aggiunto sia il percorso per raggiungere la sezione, \texttt{/reports}, al file contenente i routing privati dell'applicazione e ho creato il componente principale alla base della sezione Reports, per ora contenente solo il titolo.\\
Per permettere l'ottenimento dei dati, a lato frontend ho fatto uso di tre degli hooks che React già offre di base: \texttt{useState}, \texttt{useEffect} e \texttt{useCallback}, come nell'esempio di codice \ref{code:fetch}: \\
\begin{code}[frame=tb,label={code:fetch}, caption={Esempio di fetch dati da backend e memorizzazione in React state}]\\
1   const [data, setData] = useState([]);
2 
3   const fetchData = useCallback(async () => {
4     const resultData = await
5     /* Logica per la richiesta dei dati */
6     if (resultData) {
7       setData(resultData);
8     }
9    })
10
11  useEffect(() => {
12    fetchData();
13  }, [fetchData]);
\end{code}\\\\

\noindent \texttt{useEffect} indica che \texttt{fetchData} verrà chiamata dopo ogni rendering del componente, la quale otterrà i dati da mostrare in formato tabellare, che verranno salvati nel React state \texttt{data} tramite \texttt{useState}.
\texttt{useCallback} restituisce una funzione \texttt{fetchData} memoizzata, permettendo di non chiamarla nuovamente a ogni singolo re-rendering della pagina.\\
Per quanto riguarda la logica per la richiesta dei dati ho aggiunto un endpoint \texttt{/report} che risponde a una chiamata HTTP GET, ottenendo le \textit{attività} di un utente. Un'attività è un oggetto composto dai seguenti campi:
\begin{itemize}
  \item \texttt{id}: l'id univoco dell'attività;
  \item \texttt{user}
  \begin{itemize}
    \item \texttt{id}: l'id univoco dell'utente;
    \item \texttt{fullName}: il nome dell'utente;  
  \end{itemize}
  \item \texttt{project}
  \begin{itemize}
    \item \texttt{id}: l'id univoco del progetto a cui l'attività partiene;
    \item \texttt{name}: il nome del progetto;
  \end{itemize}
  \item \texttt{client}
  \begin{itemize}
    \item \texttt{id}: l'id univoco del cliente che commissiona il progetto;
    \item \texttt{name}: il nome del cliente;
  \end{itemize}
  \item \texttt{workspace}: il workspace dell'attività;
  \item \texttt{task}: il task che l'attività richiede;
  \item \texttt{work}: la durata in ore del'attività;
  \item \texttt{description}: breve descrizione dell'attività (opzionale);
  \item \texttt{day}: data in cui si è svolta l'attività;
  \item \texttt{workplace}: luogo in cui si è svolta l'attività.
\end{itemize}

Queste sono tutte le informazioni che la sezione report deve poter mostrare all'utente secondo i requisiti accordati.\\
L'endpoint \texttt{/reports} richiede inoltre dei query params aggiuntivi:
\begin{itemize}
  \item \texttt{from}: data prima della quale non si vuole nessuna attività;
  \item \texttt{to}: data dopo della quale non si vuole nessuna attività;
  \item \texttt{sort}: può assumere il valore 1 o -1. Nel primo caso i risultati della query saranno ordinati in modo ascendente, altrimenti in modo discendente;
  \item \texttt{token}: il token di accesso dell'utente.
\end{itemize}

A lato backend ho aggiunto la query per ottenere i dati da database.

Manca solo mostrare i dati e, a tal scopo, ho creato due nuovi React components, \texttt{ReportTable} e \texttt{ReportRow}.\\
Il primo racchiude la struttura della tabella, con tanto di header, mentre il secondo costituisce una riga della stessa. A seconda di quanti elementi saranno ottenuti dal backend, \texttt{ReportTable} chiamerà \texttt{ReportRow} un numero adeguato di volte, creando una tabella.

\subsection{Paginazone della tabella}

Ora la tabella è visibile, ma per evitare lunghi scroll verticali è necessario creare una funzionalità di paginazione, suddividendo quindi la tabella in parti più piccole, navigabili tramite una apposita interfaccia.\\
Sono possibili diverse implementazioni per una funzionalità di questo tipo, in particolare:
\begin{itemize}
  \item lato client, si richiedono tutti i dati con una sola chiamata, per poi mostrarne solo la quantità richiesta. Se l'utente vuole vedere altri dati è sufficiente prenderli da quelli già richiesti;
  \item lato server, si richiede una piccola quantità di dati, ad esempio solo 10 righe, che verranno poi mostrate. Se l'utente vuole vedere le successive 10 sarà necessario effettuare una nuova chiamata e poi mostrarle. 
\end{itemize}

In accordo con il team, ho optato per la seconda opzione in quanto si è deciso di dare priorità a non sovraccaricare troppo il client, anche perchè è facile arrivare ad avere centinaia se non migliaia di attività anche in periodi di tempo relativamente brevi. \\
Ho quindi aggiunto due React components:
\begin{itemize}
  \item \texttt{ReportTablePagination}: consiste in un menù a tendina che permette di scegliere quante righe la tabella potrà avere al massimo;
  \item \texttt{ReportTableNavigation}: consiste in un numero dinamico di bottoni che permettono le seguenti azioni:
  \begin{itemize}
    \item andare alla prima/ultima pagina;
    \item andare alla pagina precedente/successiva;
    \item andare a una pagina specifica.
  \end{itemize}
\end{itemize}

Al component principale \texttt{Report} ho aggiunto degli \texttt{useState} che permettono di implementare la funzionalità di paginazione:
\begin{itemize}
  \item \texttt{[maxRows, setMaxRows]}: controlla il numero massimo di righe impostato;
  \item \texttt{[currentPage, setCurrentPage]}: controlla la pagina in cui l'utente si trova al momento;
  \item \texttt{[totalRows, setTotalRows]}: controlla il numero totale di entry che sarebbero restituite dalla query senza paginazione.
\end{itemize}
Le prime due sono impostate dall'utente mentre per ottenere l'ultima è stato sufficiente aggiungere un campo \texttt{count} all'oggetto restituito dalla query precedentemente fatta. Sempre a questa query ho aggiunto due nuovi query params, calcolati lato client:
\begin{itemize}
  \item \texttt{skip}: quante entry "saltare" quando vado a richiedere le attività. Corrisponde a 0 se la pagina corrente è la prima, \texttt{currentPage - 1 * maxRows} altrimenti;
  \item \texttt{limit}: quante entry posso ottenere al massimo. Corrisponde a \texttt{maxRows}.
\end{itemize}

\subsection{Sprint review}
La sprint review si è conclusa in modo positivo, senza particolari correzioni necessarie. Anche i tempi sono stati rispettati.\\
La figura \ref{fig:report_table} mostra la tabella ottenuta e il relativo footer per navigarla e paginarla.

\begin{figure}[H]
	\includegraphics[width = \textwidth]{immagini/reports table.png}
	\caption{Tabella della sezione Reports}
	\label{fig:report_table}
\end{figure}             
% !TEX encoding = UTF-8
% !TEX TS-program = pdflatex
% !TEX root = ../tesi.tex

%**************************************************************
\chapter{Sprint 4}
\label{cap:sprint4}
%**************************************************************

Descrizione del quarto sprint.
             
% !TEX encoding = UTF-8
% !TEX TS-program = pdflatex
% !TEX root = ../tesi.tex

%**************************************************************
\section{Sprint 5}
\label{sec:sprint 5}
%**************************************************************

\subsection{User stories assegnate}
\paragraph{Filtro temporale (8)}\mbox{} \\[\baselineskip]
Come utente autenticato che si trova nella sezione “Reports”, voglio poter filtrare i dati visualizzati in modo che coprano solo un determinato lasso di tempo. In particolare voglio poter visualizzare i dati:
\begin{itemize}
  \item del giorno corrente;
  \item della settimana corrente;
  \item del mese corrente; 
  \item dell'anno corrente.
\end{itemize}
O, in alternativa, impostare un arco temporale personalizzato, con date d'inizio e fine a piacere.\\

\noindent Tasks:
\begin{itemize}
  \item ricerca su librerie che offrano elementi datepicker consoni ai requisiti;
  \item implementare il componente del filtro temporale;
  \begin{itemize}
    \item implementare il componente principale;
    \item implementare lo stile per il filtro;
  \end{itemize}
  \item aggiungere le opzioni per selezionare giorno/settimana/mese/anno corrente e arco temporale personalizzato;
  \item aggiornare la logica di query lato backend.
\end{itemize} 

\subsection{Filtro temporale}
Con il team abbiamo concluso che la soluzione migliore per soddisfare questo requisito fosse l'aggiunta di un semplice bottone che, cliccato, espande un calendario dove l'utente può selezionare la data d'inizio e di fine dell'arco temporale per cui vuole filtrare i dati. Inoltre, a fianco al calendario, devono essere presenti delle opzioni per impostare rapidamente gli archi di tempo predefiniti. \\
Per semplificare le cose ho fatto una breve ricerca su delle librerie che offrissero dei datepicker adatti, finendo col scegliere \texttt{react-date-range}\footcite{site:rdr}. Questa scelta è dovuta principalmente che già di suo permette di soddisfare tutte le problematiche del requisito, eccetto quella di espandere il calendario da un bottone, che comunque è risolvibile in modo piuttosto semplice. È inoltre molto intuitiva e semplice da utilizzare ed è piacevole esteticamente.\\
Per quanto riguarda il lato backend non ho dovuto fare niente poichè avevo già integrato i parametri \texttt{from} e \texttt{to} nella query al database, sapendo che sarebbero serviti.\\\\
Infine, ho aggiunto un bottone per confermare i vari filtri per limitare il numero di richieste al backend. In questo modo verrà fatta una sola richiesta una volta che l'utente ha scelto tutti i filtri, invece che una dopo la selezione di ogni filtro.

\subsection{Sprint review}
La sprint review si è conclusa in modo positivo, senza particolari correzioni necessarie. Anche i tempi sono stati rispettati.             
% !TEX encoding = UTF-8
% !TEX TS-program = pdflatex
% !TEX root = ../tesi.tex

%**************************************************************
\chapter{Sprint 6}
\label{cap:sprint6}
%**************************************************************

%**************************************************************
\section{User stories assegnate}
\paragraph{Raggruppamento dati (11)}\mbox{} \\[\baselineskip]
Come utente autenticato che si trova nella sezione “Reports”, voglio poter raggruppare i dati visualizzati per utente, cliente, progetto, task e data, con un massimo di 3 livelli di profondità.\\
Ad esempio, raggruppando per utente, la tabella mostrerà due colonne:
\begin{itemize}
  \item \texttt{Utente}, con i nomi di ogni utente;
  \item \texttt{Ore}, con la somma delle ore delle attività di ogni utente.
\end{itemize}
Se poi voglio aggiungere un secondo livello di profondità al raggruppamento, ad esempio per cliente, la tabella mostrerà sempre la colonna \texttt{Utente}, ma questa volta, sotto ad ogni nome, saranno presenti delle ulteriori righe con i nomi dei vari clienti per cui l'utente ha svolto attività, con relative somme di ore.\\

\noindent Tasks:
\begin{itemize}
  \item implementare la funzionalità di raggruppamento;
  \item aggiungere la possibilità di raggruppamenti multipli;
  \item aggiungere una nuova sezione filtri dedicata ai raggruppamenti.
\end{itemize}

\paragraph{Esportazione report in formato CSV (3)}\mbox{} \\[\baselineskip]
Come utente autenticato che si trova nella sezione “Reports”, voglio poter esportare il report correntemente mostrato in un file CSV. 
Esso dovrà contenere tutti i dati mostrati nella tabella (anche eventuali dati presenti nelle pagine non correntemente mostate).\\

\noindent Tasks:
\begin{itemize}
  \item aggiungere opzione per eseportare la tabella in formato CSV;
  \item aggiungere logica per la generazione del file CSV lato backend.
\end{itemize}


\section{Raggruppamento dati}
La prima cosa che ho fatto è stata la parte di frontend, aggiungendo 3 menù a tendina, uno per ogni livello di raggruppamento, da cui segliere il campo da raggruppare. Il secondo e terzo menù sono inizialmente nascosti e compaiono solo quando rispettivamente il primo e secondo raggruppamento sono stati scelti.\\
Lato frontend ho aggiornato la richiesta principale al backend aggiungendo 3 parametri: \texttt{groupOne}, \texttt{groupTwo} e \texttt{groupThree}. Essi sono stringhe che di default assumono il valore \texttt{none} ma, nel caso sia selezionato un raggruppamento da parte dell'utente, assumono il nome di quel raggruppamento, ad esempio \texttt{project}.\\
Ho inoltre creato un altro component per le righe "raggruppate", in modo da ottenere una struttura tale che, se l'utente ha scelto due raggruppamenti, vengano inizialmente mostrate le righe del primo raggruppamento, potendo visualizzare quelle del secondo cliccando sulla rspettiva riga. TODO: [AGGIUNGERE IMMAGINE]\\\\
Lato backend ho semplicemente aggiornato la query già presente in modo che potesse raggruppare i dati ottenuti, se richiesto. In questo caso i dati vengono ritornati con la seguente struttura:
\begin{itemize}
  \item \texttt{groupOne}
  \item \texttt{groupOneCount}
  \item \texttt{total}
\end{itemize}
Dove \texttt{groupOneCount} e \texttt{total} sono interi che rappresentano rispettivamente quanti elementi contiene il primo raggruppamento e la somma delle ore delle attività al suo interno.
\texttt{groupOne} è invece un array contenenti oggetti con la seguente struttura:
\begin{itemize}
  \item \texttt{name}: stringa contenente il nome della attività;
  \item \texttt{hours}: numero che rappresenta le ore dedicate all'attivvità;
  \item \texttt{groupTwo}: array contenente oggetti uguali a \texttt{groupOne}, eccetto relativi al secondo raggruppamento, più un campo \texttt{groupThree}.
\end{itemize}

\section{Esportazione report in formato CSV}
L'esportazione in formato CSV è risultata abbastanza rapida in quanto comportava solo ottenere i contenuti singoli dei campi richiesti dal frontend e porre tra di loro il carattere di separazione "\texttt{,}".\\
L'unica cosa a cui prestare attenzione è la eventuale presenza di virgole o virgolette nei campi che possano interferire col processo di separare i campi.
Per risolvere il problema basta circondare ogni campo tra virgolette.\\
Poi, tramite la funzione \texttt{join()} ho unito tutti i campi di una riga ponendo fra loro il carattere \texttt{,} e separato le righe con il carattere di a capo \texttt{\textbackslash n}.\\
Lato frontend ho poi aggiunto un bottone che permette di scaricare il file CSV così creato.

\section{Sprint review}
La sprint review si è conclusa in modo positivo, ma è emerso un piccolo bug nella selezione dei raggruppamenti: selezionando primo, secondo e terzo raggruppamento e poi togliendo il secondo o il primo, il terzo raggruppamento rimaneva selezionato. La sistemazione di questo bug è rimasta assegnata per lo sprint 7.             
% !TEX encoding = UTF-8
% !TEX TS-program = pdflatex
% !TEX root = ../tesi.tex

%**************************************************************
\chapter{Sprint 7}
\label{cap:sprint7}
%**************************************************************

%**************************************************************
Descrizione del settimo sprint.
% !TEX encoding = UTF-8
% !TEX TS-program = pdflatex
% !TEX root = ../tesi.tex

%**************************************************************
\chapter{Sprint 8}
\label{cap:sprint8}
%**************************************************************

%**************************************************************
\section{User stories assegnate}
\paragraph{Visualizzazione grafico (8)} \mbox{} \\[\baselineskip]
Come utente autenticato che si trova nella sezione “Reports”, voglio poter vedere un istogramma che rappresenti graficamente i dati contenuti nella tabella.\\

\noindent Tasks:
\begin{itemize}
  \item effettuare una ricerca su quale libreria Javascript possa essere più consona per creare il grafico richiesto;
  \item implementare il grafico.
\end{itemize}

\section{Visualizzazione grafico}
Per quanto riguarda la ricerca ho fatto un foglio Google simile a quello già fatto per i datepicker in cui ho raggruppato le librerie che mi sembravano migliori per svolgere l'incarico assegnato:

\begin{itemize}
  \item Chart.js\footcite{site:chartjs}
  \item D3.js\footcite{site:d3}
  \item nivo\footcite{site:nivo}
\end{itemize}

\noindent Chart.js è, tra queste, la libreria più utilizzata e spicca per la sua semplicità di utilizzo e le molteplici tipologie di grafici preconfigurate. Anche il fatto che sia popolare è un punto a favore in quanto dispone di una community più ampia, che porta a trovare con più facilità soluzioni a eventuali problemi riscontrati.\\
Anche D3.js è una libreria piuttosto popolare e non è esclusivamente dedicata alla realizzazione di grafici ma, più in generale, alla manipolazione di elementi del DOM. È caratterizzata da avere un numero altissimo di funzionalità, che si traduce nella capacità di creare grafici complessi e per obiettivi specifici, al prezzo di avere una curva di apprendimento piuttosto ripida.\\
Nivo è basato su D3.js ma offre una buona riduzione di complessità rispetto a quest'ultimo, essendo pensato esclusivamente per la creazione di grafici.\\

La scelta finale è ricaduta su Chart.js in quanto il grafico da produrre era molto semplice e, essendo l'ultima settimana, non ci sarebbe stato tempo per familiarizzare con uno strumento complesso come D3.js.

L'implementazione del grafico si è rivelata infatti piuttosto semplice, dovendo solo dichiarare delle variabili per i dati da mostrare e per delle opzioni aggiuntive del grafico:

\begin{code}[frame=tb,title={Esempio di utilizzo di Chart.js}]
1   /*Opzioni aggiuntive del grafico*/
2   const options = {
3     responsive: true,
4     scales: {
5       y: {
6         min: 0,
7       }
8     }
9   };
10
11  /*Dichiarazione dati e relative labels*/
12  const data = {
13    labels: getLabels(),
14    datasets: [
15      {
16        data: getData(),
17      }
18    ]
19  };
20
21  return <Bar options={options} data={data} />;
\end{code}\\\\

\noindent Anche la struttura di come venivano restituiti i dati non è risultata problematica nè per i visualizzare i dati raggruppati nè per quelli non raggruppati.

\section{Sprint review}
Essendo la sprint review finale ho dimostrato brevemente le funzionalità sviluppate fino ad ora, ottenendo, anche per quanto riguarda il miglioramento estetico al file PDF assegnatomi lo scorso sprint, un feedback positivo. 
% !TEX encoding = UTF-8
% !TEX TS-program = pdflatex
% !TEX root = ../tesi.tex

%**************************************************************
\chapter{Conclusioni}
\label{cap:conclusioni}
%**************************************************************

%**************************************************************
\section{Il risultato finale}
Al termine di queste otto settimane ho portato a termine una sezione di una web app capabile di mostare all'utente dati relativi alle attività aziendali in formato sia tabellare che grafico e di esportarli.\\
Questi dati sono inoltre organizzabili in molti modi tramite le opzioni di filtro e di raggruppamento.\\
\clearpage
\begin{figure}[H]
	\includegraphics[width = \textwidth]{immagini/full reports.png}
	\caption{L'intera sezione Reports}
	\label{fig:report_full}
\end{figure}


\section{Soddisfazione dei requisiti}
La tabella seguente riporta lo stato di completamento dei requisiti elencati in \S\ref{sec:requisiti}.

\begin{table}[h]
	\centering
  \rowcolors{2}{gray!25}{white}
	\begin{tabularx}{\textwidth}{X|c|c}
    \rowcolor{white}
    \textbf{Requisito} & \textbf{Importanza} & \textbf{Stato} \\
    \hline
    \makecell[l]{Apprendimento delle tecnologie di sviluppo\\come React e Node.js e per il versionamento\\del progetto come git} & Obbligatorio & Rispettato \\
    \makecell[l]{Gestione filtri avanzati di ricerca delle attività} & Obbligatorio & Rispettato \\
    \makecell[l]{Visualizzazione a tabella delle attività filtrate} & Obbligatorio & Rispettato \\
    \makecell[l]{Generazione file CSV delle attività filtrate} & Obbligatorio & Rispettato \\
    \makecell[l]{Generazione file PDF delle attività filtrate} & Desiderabile & Rispettato \\
    \makecell[l]{Visualizzazione attività tramite grafico} & Desiderabile & Rispettato \\
    \makecell[l]{Salvataggio preset filtri per ricerche future} & Desiderabile & Non Rispettato \\
    \makecell[l]{Visualizzazione widget laterale con statistiche\\ relative all'utente} & Facoltativo & Non Rispettato \\
    \makecell[l]{Gestione responsive della piattaforma} & Facoltativo & \makecell{Parzialmente\\ rispettato} \\
	\end{tabularx}
	\vspace{5pt}
	\caption{Tabella dello stato di completamento dei requisiti}
	\label{tab:raggiungimento-obiettivi}
\end{table}

\noindent Seppure tutti i requisiti obbligatori sono stati tutti rispettati, ciò non è vero per alcuni desiderabili e facoltativi. Tale mancanza è principalmente dovuta al fatto che, durante lo stage, sono sorti altri requisiti che hanno avuto la precedenza su quelli non fatti (in particolare l'inclusione e l'esclusione delle colonne e il raggruppamento dati) e alla fine sono rimasti incompiuti. Ho inoltre segnato la gestione responsive della piattaforma come parzialmente rispettato poichè, andando avanti col progetto, ho gestito il responsive dei componenti che ho implementato, ma non c'è stato tempo di fare lo stesso per quelli rimanenti. 
\section{Valutazioni personali}
Questa esperienza di stage è stata un'ottima opportunità per mettere in pratica ciò che ho appreso durante il mio percorso di laurea.
In particolare da questa esperienza mi sono rimasti più di tutti questi insegnamenti:
\begin{itemize}
  \item il lavorare in un ambiente aziendale, seguendo  e rispettando dinamiche e metodologie ben definite e come influiscano positivamente sul lavoro;
  \item ho consolidato e arricchito le mie conoscenze sullo sviluppo di applicazioni web;
  \item ho imparato a utilizzare tecnologie che non avevo mai toccato prima d'ora e pratiche che avevo poco approfondito;
  \item ho imparato l'importanza della comunicazione e del confronto con i colleghi per collaborare alla risoluzione di un problema o al raggiungimento di un obiettivo, ma, allo stesso tempo, l'importanza di consultare la documentazione ufficiale per superare un ostacolo all'apparenza invalicabile.
\end{itemize}

\noindent Infine, voglio anche dire che questa è stata un'esperienza non solo utile ma anche piacevole, e che sono contento di finire il mio percorso universitario con questa opportunità.

%**************************************************************
% Materiale finale
%**************************************************************
\backmatter
\input{inizio-fine/bibliografia.tex}

\end{document}
