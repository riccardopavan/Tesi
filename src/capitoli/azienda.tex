\chapter{L'azienda}
\label{cap:azienda}

\section{Descrizione generale}

\begin{center}
	\includegraphics[height = 4cm]{wavelop-logo.png}
\end{center}

\noindent Wavelop S.R.L.S. è una giovane azienda nata nel 2018 con sede a Treviso che si occupa di consulenza e di progetti web e mobile per piccole e medie imprese.\\
L'azienda conta circa 10 dipendenti e offre un ambiente lavorativo giovane e dinamico. Predilige inoltre lo smartworking e la flessibilità degli orari.\\
Dal 2020 propone progetti per studenti universitari che siano interessati in un tirocinio presso la loro azienda. 

(progetti anche per clienti reali, inserimento in un team di lavoro)

\section{Modello di svluppo}

Wavelop segue un modello di sviluppo \textit{agile} con metodologia \textit{scrum}. Di seguito le caratteristiche che contraddistinguono questo metodo di lavoro:

\begin{itemize}
  \item il lavoro viene diviso in archi temporali della durata di una settimana, denominati \textit{sprint};
  \item all'inizio di ogni \textit{sprint} si tiene lo \textit{sprint planning}, una riunione in cui si chiarisce quali obiettivi, espressi sotto forma di \textit{user stories}, si mira a portare a termine;
  \item alla fine di ogni \textit{sprint} si tiene invece lo \textit{sprint review}, una riunione in cui si discute cosa si è fatto durante lo \textit{sprint}, mostrando i risultati ottenuti, anche tramite brevi demo anche ai clienti;
  \item all'inizio di ogni giornata lavorativa si partecipa allo \textit{stand-up meeting}, una breve riunione della durata di circa 15 minuti, dove, a turno, ogni membro del team espone brevemente cosa ha fatto il giorno precedente e cosa intende fare oggi, oltre che mettere al corrente tutti di eventuali ostacoli incontrati nello sviluppo.
\end{itemize}

Adottare una metodologia di questo tipo permette una comunicazione costante fra i membri del team e i clienti che porta a un tracciamento chiaro e preciso dell'avanzamento del progetto e a una messa in luce immediata di eventuali incomprensioni, permettendo quindi delle correzioni tempestive, in accordo col cliente.