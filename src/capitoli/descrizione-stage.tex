\chapter{Descrizione dello stage}
\label{cap:azienda}

\section{L'azienda}

\begin{center}
	\includegraphics[height = 4cm]{wavelop-logo.png}
\end{center}

\noindent Wavelop S.R.L.S. è un'azienda nata nel 2018 con sede a Treviso che si occupa di consulenza e di progetti web e mobile per piccole e medie imprese e che conta circa 10 dipendenti.\\
Dal 2020 propone progetti per studenti universitari interessati a un tirocinio formativo che verranno poi  inseriti all'interno di un team di lavoro presso l'azienda.

\section{Modello di svluppo}

Wavelop segue un modello di sviluppo \textit{agile} con metodologia \textit{scrum}. Di seguito le caratteristiche che contraddistinguono questo metodo di lavoro:

\begin{itemize}
  \item il lavoro viene diviso in archi temporali della durata di una settimana, denominati \textit{sprint};
  \item all'inizio di ogni \textit{sprint} si tiene lo \textit{sprint planning}, una riunione in cui si chiarisce quali obiettivi, espressi sotto forma di \textit{user stories}, si mira a portare a termine;
  \item alla conclusione di ogni \textit{sprint} si tiene invece lo \textit{sprint review}, una riunione in cui si discute cosa si è fatto durante lo \textit{sprint}, mostrando i risultati ottenuti, di solito tramite brevi demo;
  \item l'attività giornaliera inizia con la partecipazione allo \textit{stand-up meeting}, una breve riunione della durata di circa 15 minuti, dove, a turno, ogni membro del team espone brevemente cosa ha fatto il giorno precedente e cosa intende fare oggi, oltre che mettere al corrente tutti di eventuali ostacoli incontrati nello sviluppo.
\end{itemize}

\noindent Adottare una metodologia di questo tipo permette una comunicazione costante fra i membri del team e i clienti che porta a un tracciamento chiaro e preciso dell'avanzamento del progetto, mettendo subito in evidenza eventuali incomprensioni, permettendo quindi delle correzioni tempestive, in accordo col cliente.

\section{Il progetto di stage}
\label{sec:progettostage}
L'azienda ha già iniziato a sviluppare un'applcazione web, denominata \emph{Timesheet}, destinata a uso interno, al fine di rendicontare e pianificare le attività dei vari componenti dell'azienda. \\
In particolare è già implementata la funzionalità di inserire attività singole nel database. \\
Manca, invece, la possibilità di mostrare, filtrandole a seconda della volontà dell'utente, le varie attività memorizzate, ottenendo quindi dati utili come la quantità di ore dedicate a un cliente in un determinato arco di tempo. \\
Queste funzionalità dovranno essere raggruppate nella sezione \emph{Reports} dell'applicazione. \\
Lo scopo finale del progetto è quindi quello di completare la sezione \emph{Reports}, sia per quanto riguarda la componente di frontend che per quella di backend.

\section{Requisiti}
\label{sec:requisiti}
Nel piano di lavoro sono stati redatti requisiti specifici che dovranno essere soddisfatti nel periodo di stage. Essi sono categorizzati in base alla importanza che ricoprono:


\paragraph{Obbligatori} Requisiti ad alta priorità, la cui importanza è primaria per la riuscita del progetto:

\begin{itemize}
  \item apprendimento delle tecnologie di sviluppo come React e Node.js e per il versionamento del progetto come git;
  \item gestione filtri avanzati di ricerca delle attività;
  \item visualizzazione a tabella delle attività filtrate;
  \item generazione file CSV delle attività filtrate.
\end{itemize}

\paragraph{Desiderabili} Requisiti a media priorità, non necessari per il completamento dello stage ma che comunque, se soddisfatti, contribuiscono notevolmente alla buona riuscita del progetto: 

\begin{itemize}
  \item generazione file PDF delle attività filtrate;
  \item visualizzazione attività tramite grafico;
  \item salvataggio preset filtri per ricerche future.
\end{itemize} 

\paragraph{Facoltativi} Requisiti a bassa priorità, portano un valore aggiunto allo stage:

\begin{itemize}
  \item visualizzazione widget laterale con statistiche relative all'utente;
  \item gestione responsive della piattaforma.
\end{itemize}

\section{Pianificazione}

Lo stage ha avuto luogo nel periodo di tempo dal 25 Luglio 2022 al 16 Settembre 2022, per un totale di 312 ore.
Sono state inoltre definite delle attività specifiche per ogni settimana di stage, rappresentate di seguito:

\paragraph{Prima settimana - Dal 25 al 29 Luglio}

\begin{itemize}
  \item discussione requisiti relativi al sistema da sviluppare con le persone coinvolte nel progetto;
  \item introduzione alla cultura aziendale;
  \item formazione sulle tecnologie adottate;
  \item prendere confidenza con la struttura già esistente e assegnazione degli strumenti necessari;
  \item analisi dei requisiti.
\end{itemize}

\noindent\textbf{Seconda settimana - Dall'1 al 5 Agosto}

\begin{itemize}
  \item analisi dei requisiti;
  \item progettazione architetturale.
\end{itemize}

\noindent\textbf{Terza settimana - Dall'8 al 13 Agosto}

\begin{itemize}
  \item progettazione architetturale;
  \item analisi e definizione user stories per lo sprint successivo.
\end{itemize}

\noindent\textbf{Dalla quarta alla ottava settimana - Dal 16 Agosto al 16 Settembre}

\begin{itemize}
  \item sviluppo delle user stories assegnate;
  \item sprint review con il referente e le altre persone coinvolte nel progetto;
  \item analisi e definizione user stories per lo sprint successivo.
\end{itemize}

\noindent Inoltre, nell'ultima settimana, è previsto un collaudo finale di ciò che è stato fatto.

\section{Tecnologie utilizzate}

È stato utilizzato lo stack tecnologico tipicamente scelto dall'azienda, composto da: \emph{Ract}, \emph{Node.js} e \emph{MongoDB}, di seguito descritte nel dettaglio, una per una.

\subsection{React}

React\footcite{site:react} è una libreria Javascript open-source component-based per creare interfacce utente mantenuta da \emph{Meta}. Adotta l'utilizzo di una sintassi che estende il linguaggio Javascript: il \emph{Javascript Synstax Extension} (JSX), che rende semplice e intuitivo creare i componenti che formano la pagina, come mostrato nell'esempio di codice \ref{code:react} .\\
\begin{code}[frame=tb, label={code:react}, caption={Esempio di utlizzo di codice JSX}]\\  
1  const name = 'Riccardo Pavan';
2
3  const ShowName = () => {
4    return (
5      <div>Mi chiamo {name}</div>
6    );
7  };

\end{code}

Un'altra particolarità di React sono gli \emph{hooks}, particolari funzioni che permettono di "agganciarsi" a varie funzionalità di \emph{lifecycle} che React offre senza dovere usare classi.

\subsection{Node.Js}

Node.Js\footcite{site:node.js} è un \emph{runtime environment} open-source basato su Javascript.\\
La sua architettura orientata agli eventi permette alle operazioni di input/output di essere asincrone ed è consona soprattutto per applicazioni scalabili e che necessitano di un grande \emph{throughput} .\\
L'utilizzo di Node.Js permette di utilizzare Javascript anche lato server, permettendo quindi di usarlo come unico linguaggio nello sviluppo di un'applicazione web, dando vita al paradigma detto \emph{Javascript everywhere}.\\

\subsection{MongoDB}

MongoDB\footcite{site:mongo} è un \emph{database management system} open-source non relazionale, orientato ai documenti, questi ultimi in formato BSON, simile a JSON.\\
La struttura non relazionale permette una maggiore libertà nello strutturare il database, utile quando si ha a che fare con molti dati molto diversi fra loro e che tendono a cambiare nel tempo.