% !TEX encoding = UTF-8
% !TEX TS-program = pdflatex
% !TEX root = ../tesi.tex

%**************************************************************
\chapter{Sprint 1}
\label{cap:sprint1}
%**************************************************************

Il primo sprint è stato dedicato principalmente ad ambientarsi al clima aziendale e alle tecnologie che verranno utilizzate più avanti. \\
Ho quindi conosciuto gli altri membri del team e mi è stato fornito materiale da studiare inerente ai seguenti argomenti:
\begin{itemize}
  \item metodologia agile e SCRUM, informazioni su termini come user stories ed epics;
  \item workflow da utilizzare nel repository condiviso, convenzioni su nomine dei commit;
  \item riunioni di routine da svolgere quotidianamente o settimanalmente;
  \item Javascript, React, CSS, Node.JS, MongoDB e porgrammazione asincrona; 
\end{itemize}

\noindent Tutta la documentazione fornita si è rivelata essere pienamente comprensiva e utile, soprattutto per quanto riguarda le tecnologie, buona parte delle quali non avevo mai usato. \\\\
Oltre a formarmi sugli argomenti precedentemente elencati, ho anche impostato l'ambiente di lavoro necessario sia a far girare la app che a svilupparla. \\
In particolare ho utilizzato i seguenti strumenti:
\begin{itemize}
  \item Docker, per poter utilizzare la immagine del database della applicazione;
  \item Studio 3T, una GUI per interfacciarsi a database MongoDB.
\end{itemize}

\noindent Tramite riga di comando potevo poi far girare sia la parte client che la parte server in locale, permettendomi di modificare codice e avere feedback visivo immediato. \\\\
\noindent Infine ho redatto una bozza di un documento atto a elencare le user stories per avere un'idea più chiara di ciò che avrei dovuto fare negli sprint successivi.\\
Seppure il docmento non fosse ancora completo è risultato subito chiaro che sarebbe stata necessaria una libreria per creare grafici. Ho quindi dedicato l'ultimo giorno della settimana a ricercare delle librerie Javascript per grafici, evidenziandone vantaggi e difetti per potere poi giungere a una decisione con gli altri membri del team.