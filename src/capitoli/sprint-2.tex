% !TEX encoding = UTF-8
% !TEX TS-program = pdflatex
% !TEX root = ../tesi.tex

%**************************************************************
\chapter{Sprint 2}
\label{cap:sprint2}
%**************************************************************

Durante il secondo sprint ho redatto la versione conclusiva del documento relativo alle user stories, approvato poi dal mio relatore esterno.

\noindent Ogni user story del documento presenta i seguenti elementi:
\begin{itemize}
  \item nome della useer story;
  \item un numero intero positivo che indica il "peso" in termini temporali che ci si aspetta che la user story abbia. Questi numeri seguono l'andamento della serie di Fibonacci e sono stati utli per allocare meglio le varie user stories all'interno degli sprint; 
  \item breve descrizione di ciò che si deve ottenere una volta conclusa la user story, dal punto di vista dell'utente;
  \item elenco di task, suddivisi tra frontend e backend, che dicono in modo più speccifico e tecnico cosa fare per concludere la user story;
\end{itemize}

\noindent All'inizio di ogni capitolo relativo a uno sprint indicherò le user stories su cui andrò a lavorare.

Ho inoltre creato dei diagrammi UML dei casi d'uso. (Li metto qua?)

Per concludere la settimana ho lavorato a dei piccoli bugfix e features, anche per familiarizzare con la web app e i pattern utilizzati.
In particolare ho: 
\begin{itemize}
  \item sistemato un problema con la immagine nella pagina di login, che non si rimpiccioliva nel caso la finestra facesse lo stesso;
  \item fatto sì che, nella pagina utente, nel caso quest'ultimo non la avesse inserita, venissero mostate le iniziali dell'utente invece che nulla;
  \item implementato una finestra di dialogo che semplicemente chiede di confermare o annullare un'azione.
\end{itemize}

Seppure i primi due problemi fossero molto semplici e risolvibili principalmente con del codice CSS, creare la finestra di dialogo mi è stato utile sia a familiarizzare con React, che non avevo mai utilizzato, ma anche a capire meglio come fosse strutturata la applicazione.